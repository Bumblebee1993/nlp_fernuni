\section{Zeichen und Listen}

Einige Beispiele

\begin{itemize}
\item \textit{Griechische Buchstaben:} $\alpha, \beta, \gamma, \dots, \phi,
\psi$ und große $\Gamma, \Delta, \dots, \Phi, \Psi$

\item \textbf{Einige Zeichen:} $\cap, \cup, \subseteq, \setminus,
  \times, \otimes, \le, \ge, \in, \parallel, \perp, \angle$
\end{itemize}
und eine nummerierte Liste
\begin{enumerate}
\item in der 
\item aber
\item nichts steht
  \begin{enumerate}
  \item Dennoch hat sie 
  \item eine Unterliste
  \end{enumerate}
\item 
  \begin{itemize}
  \item Die nicht nummeriert sein muss
    \begin{itemize}
    \item und ihrerseits Unterlisten haben kann
    \item usw.
    \end{itemize}
  \end{itemize}
\end{enumerate}

Wenn man (wie hier) das Packet ``enumerate'' geladen hat, kann man die
Ausgestaltung der Nummerierung leichter gestalten.

\begin{enumerate}[\bfseries (E1)]
\item M\"oglich sind
  \begin{enumerate}[A.)]
  \item 1
  \item a
  \item A
  \item i
  \item I
  \end{enumerate}
  \item jeweils mit Klammern, Punkten, Kommata usw.
\end{enumerate}

\begin{description}
\item [Test] hier kommt der Text
\item[Test II] nächster Punkt
\item[Item] 
\end{description}