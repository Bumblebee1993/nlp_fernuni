\section{Text ...}

%\leavevmode\marginpar{Test test test}
... wird einfach eingetippt. Leerzeichen trennen wie üblich Wörter, die Anzahl ist egal.
Auch Zeilenumbrüche im Quelltext werden ignoriert.
Zeilenumbrüche macht das System
selbständig. Ein neuer Absatz entsteht, wenn man im Quelltext eine

Zeile frei lässt.  



Um Wortabstände und Zeilenabstände künstlich zu verändern gibt es die Befehle
$\backslash$hspace \hspace{9mm} (siehe Quelltext) und $\backslash$vspace \vspace{10mm}
die man aber selten braucht. Man beachte, dass der zweite Befehl erst beim folgenden Zeilenumbruch wirkt. Er wird eigentlich nur zwischen

\vspace{10mm}
Absätzen und im Mathematiksatz eingesetzt.\\  % neue Zeile ohne Absatz
\textbf{Empfehlung:} Sparsam (oder gar nicht) benutzen!

\subsection{Schriftgroeße}


Man kann die Größe der Buchstaben ändern:

{\large groß}, {\LARGE größer}, {\huge noch größer} und {\small klein}
oder {\tiny winzig} 

\subsection{Schriftart}


Um im Text Teil hervorzuheben kann man 

\textit{kursive Schrift}
\quad % = Abstand ca. 3x Leerzeichen
\textbf{fett}
\quad
\texttt{Schreibmaschinen-Schrift} 
\quad
\textsc{Kapitälchen}

benutzen.

Beispiel für Tabulatoren:

\begin{tabbing}
Wenn \= es regnet \\
\> dann \= zieh Schuhe an,\\
\> \> einen Hut;\\
\> oder lächle. \\
Verlasse dann das Haus.
\end{tabbing}

$\backslash\backslash$ markiert das Ende\\ der Zeile.

\paragraph{Farben}
Dazu braucht man aber die Definitionen aus der Präambel.

\newpage
\subsection{Tabelle}

Hier ein Beispiel einer Tabelle:

\begin{table}[!htbp]
%\begin{center}
\caption{Eine Tabelle mit Inhalt}
 \begin{tabular}{|l|l|l|p{0.3\linewidth}|}      
    \hline
    Day & Min Temp & Max Temp & Summary \\ \hline
    Monday & 11C & 22C & A clear day with lots of sunshine.
    However, the strong breeze will bring down the temperatures. \\ \hline
    Tuesday & 9C & 19C & Cloudy with rain, across many northern regions. Clear spells 
    across most of Scotland and Northern Ireland, 
    but rain reaching the far northwest. \\ \hline
    Wednesday & 10C & 21C & Rain will still linger for the morning. 
    Conditions will improve by early afternoon and continue 
    throughout the evening. \\
    \hline
    \end{tabular}
    
%\end{center}

\end{table}

\begin{table}[!htbp]
%\begin{center}
\caption{Eine Tabelle mit Inhalt}
 \begin{tabular}{|l|l|l|p{0.3\linewidth}|}      
    \hline
    Day & Min Temp & Max Temp & Summary \\ \hline
    Monday & 11C & 22C & A clear day with lots of sunshine.
    However, the strong breeze will bring down the temperatures. \\ \hline
    Tuesday & 9C & 19C & Cloudy with rain, across many northern regions. Clear spells 
    across most of Scotland and Northern Ireland, 
    but rain reaching the far northwest. \\ \hline
    Wednesday & 10C & 21C & Rain will still linger for the morning. 
    Conditions will improve by early afternoon and continue 
    throughout the evening. \\
    \hline
    \end{tabular}
    
%\end{center}

\end{table}

%\addcontentsline{lot}{table}{Tabelle}
%\paragraph{Kommentare zur Tabelle:}
%$\{$lr|c$\}$ bedeutet, dass die erste Spalte linksbündig, die zweite
%Spalte rechtsbündig, dann eine Linie und die dritte Spalte zentriert
%gesetzt wird.

