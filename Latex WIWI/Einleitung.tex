\section{Einleitung}

%\leavevmode\
\textAndNote{In dieser Datei sind einige Beispiele für Anwendungen von LaTeX\footnotemark enthalten. Es handelt sich aber keineswegs um eine echte Einführung.}{Ein wenig Text für eine eingerückte Marginalie.}Einziges Ziel ist es eine Vorlage zu liefern, die als Gerüst für eine in LaTeX verfasste Ausarbeitung\footnotemark dienen kann. Dazu finden sich Beispiele für Formatierungsbefehle und den Mathematiksatz. 


%\footnote{} kann auf Seiten benutzt werden, auf denen es keine minipages gibt.

Das Zeichen \% % Beispiel
im Quelltext markiert einen Kommentar. Alles was hinter diesem Zeichen in derselben Zeile steht, wird beim \"ubersetzen ignoriert.

Befehle beginnen immer mit $\backslash$, etwa $\backslash$hspace,
Argumente werden mit $\{$ und $\}$ geklammert.

Jede LaTeX-Datei beginnt mit einer Prämbel in der nur Befehle
eingegeben werden können. Die Bedeutung der benutzten Befehle in
dieser Datei ist in der LaTeX-Datei als Kommentar kurz angedeutet.

\begin{itemize}
\item $\backslash$begin$\{$document$\}$ markiert den Punkt ab dem Text
eingegeben werden kann.
\item $\backslash$end$\{$document$\}$ beendet den Text. Was nach diesem Befehl steht, wird beim übersetzen ignoriert.
\end{itemize}

Umlaute erhält man durch \"a, wenn (wie hier) das Packet ngerman
geladen ist durch "a, und wenn (wie hier) das Packet ``inputenc'' mit ``latin1'' (oder ``utf8'')geladen ist auch durch ä. Entsprechend \ss, "s, ß, "u, "o, ü, ö.\\


Wenn man einen etwas längeren Text in mehrere Abschnitte unterteilen will oder nur die nächste Zeile beginnen möchte, bietet sich dafür $\backslash$ $\backslash$(siehe Quelltext) an.\\

Dieser Text sollte nun in einem neuen Absatz beginnen.




\footnotetext[1]{Beispieltext Quellenangabe etc.}
\footnotetext[2]{Test Test 2. Fußnote}

